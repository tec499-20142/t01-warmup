%
% Portuguese-BR vertion
% 
\documentclass{article}

\usepackage{ipprocess}
% Use longtable if you want big tables to split over multiple pages.
% \usepackage{longtable}
\usepackage[utf8]{inputenc} 
% Babel package is used to translate keywors to a specific language. 
% The option "brazil" defines portuguese-brazil as default language.
% Babel is also useful for hiphenation.
\usepackage[brazil]{babel}

\sloppy

\graphicspath{{./pictures/}} % Pictures dir
\makeindex
\begin{document}

\DocumentTitle{Especificação de Requisitos}
\Project{Unidade de Operações Aritméticas}
\Organization{Universidade Estadual de Feira de Santana}
\Version{Compilação 1.0}
\capa

\newpage

%%%%%%%%%%%%%%%%%%%%%%%%%%%%%%%%%%%%%%%%%%%%%%%%%%
%% Revision History
%%%%%%%%%%%%%%%%%%%%%%%%%%%%%%%%%%%%%%%%%%%%%%%%%%
\section*{\center Histórico de Revisões}
  \vspace*{1cm}
  \begin{table}[ht]
    \centering
    \begin{tabular}[pos]{|m{2cm} | m{7.2cm} | m{3.8cm}|} 
      \hline
      \cellcolor[gray]{0.9}
      \textbf{Data} & \cellcolor[gray]{0.9}\textbf{Descrição} & \cellcolor[gray]{0.9}\textbf{Autor(s)}\\ \hline
      \hline
      \small 11/08/2014 & \small Concepção do documento & \small joaocarlos \\ \hline      
      \small 12/08/2014 &
      \begin{small}
        \begin{itemize}
          \item Inclusão do requisito [FR2.12];
          \item Substituição do requisito [FR2.22];
          \item Remoção do requisito [FR2.43];
          \item Inclusão do requisito [NFR1];
        \end{itemize}
      \end{small} & \small joaocarlos e x-afn \\ \hline 
      \small 18/08/2014 & \small Atualização do modelo & \small joaocarlos \\ \hline       
	  \small 16/09/2014 & \small Concepção da introdução e revisão do documento & \small manuellemacedo \\ \hline    
    \end{tabular}
  \end{table}

\newpage

% TOC instantiation
\tableofcontents
\newpage

%%%%%%%%%%%%%%%%%%%%%%%%%%%%%%%%%%%%%%%%%%%%%%%%%%
%% Document main content
%%%%%%%%%%%%%%%%%%%%%%%%%%%%%%%%%%%%%%%%%%%%%%%%%%
\section{Introdução}
Um \textit{IP-Core} é um circuito que tem como principal objetivo o reuso para diversos propósitos. Grandes empresas, hoje em dia, buscam o reuso para economia de tempo e dinheiro porque ao adquirir um \textit{IP-Core} ele tem em mãos um circuito devidamente testado e pronto para o reuso em seu próprio projeto.  Neste documento será tratado sobre um \textit{IP-Core} que tem como funcionalidade realizar operações logicas e aritméticas utilizando a interface de comunicação RS-232, este poderá ser usado para diversos propósitos.    


\subsection{Visão Geral do Documento}
  \begin{itemize}
   \item \textbf{Requisitos funcionais -} lista de todos os requisitos funcionais.
   \item \textbf{Requisitos não funcionais -} lista de todos os requisitos não funcionais.
   \item \textbf{Dependências -} conjunto de dependências de IP-cores previstos.
   % \item \textbf{Notas -} apresenta a lista de notas apresentadas ao longo do documento.
   % \item \textbf{Referências -} lista de todos os textos referenciados nesse documento.
  \end{itemize}

  \subsection{Definições}
  
  \FloatBarrier
  \begin{table}[H]
    \begin{center}
      \begin{tabular}[pos]{|m{2.5cm} | m{11.5cm}|} 
        \hline
        \cellcolor[gray]{0.9}\textbf{Termo} & \cellcolor[gray]{0.9}\textbf{Descrição} \\ \hline
        \flushright Requisito Funcional     & Requisitos de hardware que compõem os módulos, descrevendo as ações que o 
                                    mesmo deve estar apto a executar. Estas informações são capturadas a partir 
                                    do desenvolvimento dos casos de uso, que documentam as entradas, os processos 
                                    e as saídas geradas.  \\ \hline
        \flushright Requisito Não \mbox{Funcional} & Requisitos de hardware que compõem os módulos, representando as características 
                                    que o mesmo deve ter, ou restrições que o mesmo deve operar. Estas características
                                    referem-se a técnicas, algoritmos, tecnologias e especificidades do Sistema como um todo.  \\ \hline
        \flushright Dependências              & Requisitos de reuso de IP-cores, descrevendo as funções que cada um deve exercer. \\ \hline
      \end{tabular}
    \end{center}
  \end{table}  

  % inicio da tabela de acronimos e abreviacoes do documento
  \subsection{Acrônimos e Abreviações}
    \FloatBarrier
    \begin{table}[H]
      \begin{center}
        \begin{tabular}[pos]{|m{2cm} | m{12cm}|} 
          \hline
          \cellcolor[gray]{0.9}\textbf{Sigla} & \cellcolor[gray]{0.9}\textbf{Descrição} \\ \hline
          FR      & Requisito Funcional  \\ \hline
          NFR     & Requisito Não Funcional  \\ \hline
          D       & Dependência  \\ \hline
        \end{tabular}
      \end{center}
    \end{table}  
  % fim  

  % inicio da descriao de prioridades de requisitos
  \subsection{Prioridades dos Requisitos}
    \FloatBarrier
    \begin{table}[H]
      \begin{center}
        \begin{tabular}[pos]{|m{2cm} | m{12cm}|} 
          \hline
          \cellcolor[gray]{0.9}\textbf{Prioridade} & \cellcolor[gray]{0.9}\textbf{Característica} \\ \hline
          Importante      & Requisito sem o qual o sistema funciona, porém não como deveria.  \\ \hline
          Essencial       & Requisito que deve ser implementado para que o sistema funcione.  \\ \hline
          Desejável       & Requisito que não compromete o funcionamento do sistema.  \\ \hline
        \end{tabular}
      \end{center}
    \end{table}  
  % fim

\section{Requisitos Funcionais}

  \subsection{Interface de Comunicação RS232}
  
    \begin{functional}
      \requirement
      {Taxa de velocidade de transmissão e recepção (\textit{baud rate})}
      {O processo de comunicação seria deve ocorrer à uma taxa de transmissão igual a $9600$ bps.}
      {Essencial}

      \requirement
      {Modo de operação do controlador RS232}
      {O controlador serial RS232 opera no modo 8-N-1.}
      {Essencial}      

      \requirement
      {Frequência de operação do circuito} 
      {A frequência de clock do sistema deve ser definida de acordo com a taxa de frequência do oscilador principal do kit FPGA utilizado para prototipação.}
      {Essencial}   

      \requirement
      {Padrão de sequência de pacotes para uma operação}
      {Uma operação é representada pela transmissão de três pacotes de dados na ordem que se segue: (1) código da operação, (2) primeiro operando e (3) segundo operando.}
      {Essencial}       

    \end{functional}

  \subsection{Interface de Saída}
  
    \begin{functional}
      \requirement
      {Mecanismo de armazenamento de resultado}
      {O resultado de uma operação deve ser armazenado em um registrador interno.}
      {Importante}

      \requirement
      {Protocolo de saída de dados do resultado}
      {Os dados devem ser apresentado em sua codificação binária, representada através de um conjunto de LEDs.}
      {Essencial} 

    \end{functional}  

  \subsection{Conjunto de Operações Aritméticas} 
  
    \begin{functional}
      \requirement
      {Operação de soma}
      {O módulo deve ser capaz de realizar a operação de soma de dois valores de 8 bits.}
      {Essencial}

      \requirement
      {Operação de subtração}
      {O módulo deve ser capaz de realizar a operação de subtração de dois valores de 8 bits.}
      {Essencial}

      \requirement
      {Operação de multiplicação}
      {O módulo deve ser capaz de realizar a operação de multiplicação de dois valores de 8 bits. }
      {Prioridade: Essencial}

      \requirement
      {Operação de divisão}
      {O módulo deve ser capaz de realizar a operação de divisão de dois valores de 8 bits.}
      {Essencial} 

      \requirement
      {Tamanho da palavra de saída}
      {O componente deve apresentar uma saída única de 8 bits para todas as operações aritméticas.}
      {Importante}       

      \requirement
      {Detecção de overflow aritmético}
      {O módulo deve ser capaz de detectar \textit{overflow} aritmético.}
      {Importante}
    \end{functional}

  \subsection{Conjunto de Operações Lógicas} 
  
    \begin{functional}
      \requirement{Operação AND}
      {O módulo deve ser capaz de realizar a operação AND lógico de dois operandos de 8 bits.}
      {Essencial}

      \requirement{Operação OR}
      {O módulo deve ser capaz de realizar a operação OR lógico de dois operandos de 8 bits.}
      {Essencial}      
     
    \end{functional}    

  \section{Requisitos não Funcionais}
  Esta seção apresenta a lista de Requisitos não Funcionais do projeto.

  \begin{nonfunctional}
    \requirement{Transmissão dos dados via porta serial}
    {Os dados devem ser transmitidos através de uma interface de software dedicada.}
    {Importante}

  \end{nonfunctional}

\section{Dependências}

  \begin{dependencies}
    \dependency{Controlador RS232}
    {Módulo de comunicação serial via protocolo RS232 disponível no repositório OpenCores.}

\end{dependencies}  

% Optional bibliography section
% To use bibliograpy, first provide the ipprocess.bib file on the root folder.
% \bibliographystyle{ieeetr}
% \bibliography{ipprocess}

\end{document}
